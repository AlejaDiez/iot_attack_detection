\capitulo{2}{Objetivos del proyecto}

Este apartado explica de forma precisa y concisa cuales son los objetivos que se persiguen con la realización del proyecto. Se puede distinguir entre los objetivos marcados por los requisitos del software a construir, los objetivos de carácter técnico que plantea a la hora de llevar a la práctica el proyecto y los objetivos personales a conseguir por el alumno.

\section{Objetivos Generales}
\label{sec:ObjetivosGenerales}
\begin{itemize}
    \item Desarrollar un simulador de red de dispositivos que genere tráfico entre ellos, permita lanzar ataques y soporte la integración de modelos de inteligencia artificial basados en TensorFlow para la detección de intrusiones.
    \item Entrenamiento de un modelo de inteligencia artificial mediante un enfoque federado de entrenamiento.
\end{itemize}

\section{Objetivos Técnicos}
\label{sec:ObjetivosTecnicos}
\begin{itemize}
    \item Diseñar un simulador de red usando el framework Angular que permita generar tráfico entre dispositivos y facilite la integración de modelos de inteligencia artificial.
    \item Implementar diversos tipos de ataques en la red simulada para evaluar la capacidad de detección del sistema.
    \item Dividir un conjunto de datos entre múltiples dispositivos, representando un entorno distribuido.
    \item Entrenar un modelo de detección de ataques utilizando TensorFlow, distribuido entre dispositivos mediante el framework de aprendizaje federado Flower.
    \item Recopilar y analizar métricas relevantes durante el entrenamiento federado para evaluar el rendimiento del modelo.
    \item Detectar ataques en la red analizando los flujos de datos con el modelo entrenado.
\end{itemize}

\section{Objetivos Personales}
\label{sec:ObjetivosPersonales}
\begin{itemize}
    \item Aplicar los conocimientos adquiridos durante el Grado en Ingeniería Informática en un proyecto real y práctico.
    \item Profundizar en el uso de framework de desarrollo web (Angular) y entrenamiento de inteligencia artificial (TensorFlow y Flower).
    \item Aprender sobre entrenamientos de redes neuronales de forma distributiva e implementación en entornos reales.
    \item Uso de software de control de versiones como Git junto a herramientas de alojamiento de repositorios como GitHub.    
\end{itemize}
