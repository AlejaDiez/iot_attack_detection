\apendice{Anexo de sostenibilización curricular}

\section{Introducción}
\label{sec:IntroduccionSostenibilizacion}
En este anexo se presenta una reflexión personal sobre los aspectos de sostenibilidad considerados durante el desarrollo del proyecto, centrado en la simulación y detección de ataques sobre una red mediante técnicas de inteligencia artificial.

Este proyecto no solo propone solucionar problemas tecnológicos, sino que también tiene como perspectiva una visión de sostenibilidad, tanto en temas sociales, económicos y ambientales.

\section{Reflexión sobre la sostenibilidad en el proyecto}
\label{sec:Reflexion}
\subsection{Sostenibilidad social}
\label{subsec:Social}
El desarrollo de este proyecto contribuye a la sostenibilidad social al centrarse en la mejora de la seguridad en internet, un aspecto de gran importancia para proteger la privacidad y derechos de los usuarios. Esta capacidad de detectar ataques permite reducir el riesgo de colapso en organizaciones de uso común como la sanidad, educación, servicios públicos...

Además, al enfocar el entrenamiento de la red neuronal mediante el aprendizaje federado, se evita la necesidad de centralizar datos utilizados para el entrenamiento del modelo. Esto supone una reducción significativa de riesgos asociados a la privacidad de los usuarios, cumpliendo con normativas de protección de datos.

\subsection{Sostenibilidad económica}
\label{subsec:Economica}
Desde una visión económica, el proyecto aporta mucho valor a mejorar la eficiencia de detección de ataques, logrando reducir costes asociados a diferentes problemas derivados de estos incidentes, como la interrupción de servicios o la pérdida de datos. La detección temprana de los ataques ayuda a reducir los riesgos económicos de una empresa o institución.

\subsection{Sostenibilidad ambiental}
\label{subsec:Ambiental}
Gracias a las técnicas de aprendizaje automático, un modelo va a ser mucho más eficiente a la hora de detectar ataques frente a un programa tradicional. Esta mayor eficiencia permite reducir el uso innecesario de recursos, contribuyendo así a un menor consumo energético.

Además, el proyecto ofrece un simulador que permite probar diferentes modelos sin necesidad de desarrollar una infraestructura, así se puede lograr reducir significativamente el impacto ambiental y económico asociado al uso del hardware.

\section{Competencias de sostenibilidad adquiridas}
\label{sec:Competencias}
Durante el desarrollo del proyecto, he podido adquirir una serie de competencias vinculadas a la sostenibilidad, según los principios definidos por la CRUE\footnote{Conferencia de Rectores de las Universidades Españolas}.

\subsection{SOS 1 - Competencia en la contextualización crítica del conocimiento estableciendo interrelaciones con la problemática social, económica y ambiental, local y/o global}
\label{subsec:SOS1}
He logrado desarrollar la capacidad de contextualizar críticamente el uso de redes neuronales y otros métodos tecnológicos en relación con la problemática social, económica y ambiental. Puedo comprender las conexiones que hay entre los métodos aplicados en el proyecto y su impacto en el medio ambiente y la sociedad, produciendo así un desarrollo más responsable y alineado con dichos principios.

\subsection{SOS2 - Competencia en la utilización sostenible de recursos y en la prevención de impactos negativos sobre el medio natural y social}
\label{subsec:SOS2}
Durante el desarrollo del proyecto, he intentado, en la medida de lo posible, la optimización de recursos, implementando el uso de códigos eficientes y reduciendo el acceso a datos personales para el entrenamiento del modelo, preservando así la privacidad.

\subsection{SOS4 - Competencia en la aplicación de principios éticos relacionados con los valores de la sostenibilidad en los comportamientos personales y profesionales}
\label{subsec:SOS4}
He aprendido a aplicar principios éticos relacionados con valores de sostenibilidad. Este aprendizaje ha proporcionado una mayor responsabilidad por el medio ambiente, el fomento de la equidad social y una toma de decisiones que promuevan un desarrollo sostenible.

\section{Conclusiones}
\label{sec:Conclusiones}
El desarrollo del proyecto sobre aprendizaje federado y simulación de red ha sido una experiencia para aprender conocimientos tanto en temáticas tecnológicas como principios de sostenibilidad aplicados a la informática.
