\capitulo{6}{Trabajos relacionados}

Como se comentó en la introducción, la detección de ataques en redes de dispositivos es una tarea de gran relevancia en la sociedad. Cada vez más dispositivos se conectan a la red y, con ello, aumenta su vulnerabilidad frente a posibles ataques. En este contexto, surgen múltiples alternativas para hacer frente a estos problemas. Durante el desarrollo de este proyecto, se ha propuesto una estrategia para paliar esta situación.

En este apartado se exponen los distintos artículos y recursos que más han influenciado en el desarrollo del proyecto, así como el artículo académico que se escribió a partir de la investigación realizada sobre el aprendizaje federado.

\begin{itemize}
    \item \textbf{Communication-Efficient Learning of Deep Networks from Decentralized Data}: artículo que introduce el concepto de Federated Learning, además de analizar sus ventajas y desventajas. Este trabajo fue una primera toma de contacto con las bases teóricas del aprendizaje federado. El desarrollo de este proyecto se ha basado en este artículo, aplicándolo a un área más específica, como es la detección de ataques en redes. (Fuente~\cite{aprendizaje_federado_articulo})
    \item \textbf{TensorFlow Federated Tutorial Session}: conferencia que muestra el concepto de aprendizaje federado y su aplicación con la biblioteca TensorFlow Federated. Este recurso fue clave en los primeros pasos de implementación de un entorno federado, aunque después no se haya usado en la realización del TFG. (Fuente~\cite{tensorflow_federated_tutorial})
    \item \textbf{Listado de herramientas para aprendizaje federado}: blog que compara las distintas alternativas que existen en el mercado de código abierto para la implementación de aprendizaje federado. Permitió evaluar qué tecnología era la más flexible para el desarrollo del proyecto. (Fuente~\cite{open_source_frameworks})
    \item \textbf{Documentación TensorFlow}: documentación oficial de la biblioteca TensorFlow. Es la Fuente principal de ayuda en la que me he apoyado en los momentos donde han surgido problemas con el uso de la biblioteca. (Fuente~\cite{tensorflow})
    \item \textbf{Documentación TensorFlowJS}: documentación oficial de la biblioteca TensorFlowJS. Esta documentación me ha ayudado a la implementación de los modelos de inteligencia artificial en el simulador web y a la solución de los problemas que he tenido durante la implementación. (Fuente~\cite{tensorflowjs})
    \item \textbf{Documentación TensorFlow Federated}: documentación oficial de la biblioteca TensorFlow Federated. Fue la base para la implementación de las primeras pruebas realizadas sobre una arquitectura de entrenamiento federado. (Fuente~\cite{tensorflow_federated})
    \item \textbf{Documentación Flower}: documentación oficial de la biblioteca Flower. Me sirvió de gran ayuda en el desarrollo del script realizado para la implementación de un aprendizaje federado sobre un caso concreto. Tiene muchos ejemplos usando diferentes bibliotecas de entrenamiento de redes neuronales, entre las que se encuentra TensorFlow. (Fuente~\cite{flower})
    \item \textbf{Curso de Angular}: curso sobre desarrollo de aplicaciones web con el framework Angular. Aunque el curso incluye la creación de aplicaciones web, que poco tienen que ver con el simulador realizado durante el proyecto, ha sido de gran ayuda para comprender las nociones básicas del framework, así como las buenas prácticas de desarrollo. (Fuente~\cite{angular})
\end{itemize}

\section*{Federated Learning for the Detection of Attacks on IoT Devices}
\label{sec:articulo_academico}
Durante mi participación en la beca de colaboración dentro de un grupo de investigación universitario, desarrollé un trabajo centrado en el estudio y aplicación de técnicas de aprendizaje automático para la detección de ataques sobre redes de dispositivos IoT. El objetivo principal del proyecto fue adaptar y entrenar un modelo ya existente utilizando un enfoque distribuido basado en aprendizaje federado.

Como resultado de este trabajo, se elaboró un artículo académico cuyo propósito fue introducir el concepto de aprendizaje federado y comparar su eficacia con la de un modelo entrenado de forma tradicional. La investigación se centró en evaluar el rendimiento de ambos modelos mediante diversas métricas, obteniendo resultados bastante buenos, especialmente teniendo en cuenta los desafíos del uso de técnicas federadas.

Para dar visibilidad al trabajo realizado, el artículo fue enviado al congreso internacional CISIS 2025 (International Conference on Computational Intelligence in Security for Information Systems), encontrándose actualmente en proceso de revisión.
