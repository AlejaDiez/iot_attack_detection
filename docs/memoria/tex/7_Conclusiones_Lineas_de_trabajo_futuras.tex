\capitulo{7}{Conclusiones y Líneas de trabajo futuras}

\section{Conclusiones}
\label{sec:conclusiones}
Durante el periodo que ha durado este proyecto, se ha desarrollado un simulador de rede de dispositivos IoT, con objetivo de ser un espacio de pruebas para distintos modelos de redes neuronales destinados a la detección de ataques. Además, se ha logrado el entrenamiento de un modelo de inteligencia artificial de forma distribuida, priorizando en todo momento la privacidad de los datos.

Se puede concluir de que el objetivo general del proyecto se ha cumplido de manera satisfactoria. Se ha logrado desarrollar un simulador básico de red de dispositivos IoT, capaz de generar un flujo de paquetes y, mediante la implementación de un modelo de red neuronal, detectar estos ataques simulados. 

En cuanto al entrenamiento de la red neuronal para la detección de ataques, aún queda un largo camino por recorrer hasta que el modelo sea plenamente funcional y efectivo. La complejidad está en la obtención de datos de calidad, dificultando el aprendizaje. Por ello, se requieren más iteraciones para mejorar su precisión y capacidad de generalización.

Una vez finalizado el proyecto, es importante destacar algunos puntos clave que he tenido en cuenta:
\begin{itemize}
    \item Tener una base teórica en los temas tratados durante el proyecto (artículos de investigación y conocimientos del grado).
    \item Tener claro que ofrece cada tecnología que he usado, para ahorrar tiempo durante el desarrollo.
    \item Una buena planificación del trabajo, con los objetivos claros y materiales necesarios.
\end{itemize}

Gracias a la realización de este proyecto, me ha permitido profundizar en diversas áreas del ámbito de la inteligencia artificial y el desarrollo de software. Para ser más concretos, he tenido la oportunidad de investigar y aplicar técnicas avanzadas de entrenamiento de redes neuronales, como el aprendizaje federado. Esta técnica me ha permitido comprender mejor la importancia de tener un buen conjunto de datos y como este es capaz de influir sobre la calidad del modelo resultante.

Asimismo, el proyecto ha sido un entorno perfecto para mejorar mis habilidades en el desarrollo de software, consolidando conocimientos sobre las herramientas que he usado.

Por otro lado, gestionar un proyecto de esta magnitud me ha llevado a mejorar mis habilidades organizativas. He aprendido a planificar tareas, establecer prioridades, gestionar tiempos y adaptarme a imprevistos que he logrado solucionar buscando alternativas.

\section{Líneas futuras}
\label{sec:lineas_futuras}
A pesar de los logros alcanzados por este proyecto, todavía quedan varios aspectos a mejorar para obtener un proyecto mas consolidado.

\begin{itemize}
    \item \textbf{Topologías de red más personalizadas}: una posible línea de mejora del simulador consiste en ampliar su capacidad para representar topologías de red más complejas y personalizadas. Sería deseable permitir la incorporación de múltiples subredes interconectadas. Asimismo, la inclusión de nuevos dispositivos de gestión (switches o puntos de acceso).
    \item \textbf{Mayor realismo en la red}: otra posible mejora para el simulador consistiría en la implementación de más protocolos u aspectos que permitan tener un entorno de red más realista.
    \item \textbf{Actualización del dataset}: una mejora significativa para incrementar el rendimiento del modelo consistiría en entrenarlo con conjuntos de datos más amplios y representativos. Se podrían emplear versiones más recientes del dataset utilizado en este proyecto.
    \item \textbf{Exploración de diferentes arquitecturas de redes neuronales}: otro aspecto a investigar sería el uso de diferentes arquitecturas de redes neuronales, para poder compara entre diferentes alternativas, cual sería la más adecuada para este proyecto.
    \item \textbf{Entorno de entrenamiento más realista}: para avanzar en la investigación del entrenamiento distribuido, una posible línea de trabajo futura consistiría en simular un entorno más cercano a escenarios reales. Esto implicaría entrenar el modelo con un mayor número de clientes, distribuidos en varias subredes. Esto permitiría evaluar el comportamiento del sistema en condiciones de mayor escala y diferencias. Además, sería interesante estudiar el impacto del uso de datos no independientes e idénticamente distribuidos en el rendimiento del modelo. Por último, también se podría analizar las implicaciones que lleva el entrenamiento del modelo en la propia red, en consumo de ancho de banda, latencia o congestión.
\end{itemize}