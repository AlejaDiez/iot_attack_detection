\apendice{Especificación de Requisitos}

\section{Introducción}
\label{sec:RequisitosIntroduccion}
Esta sección ofrece una visión general del proyecto, incluyendo su propósito, público objetivo y uso previsto del software desarrollado.

\subsection{Propósito del producto}
\label{subsec:Proposito}
El propósito de este software es proporcionar un entorno de pruebas para diferentes modelos de inteligencia artificial orientados a la detección de ataques. La herramienta permite a los usuarios diseñar diferentes tipologías de red, y evaluar los modelos desarrollados para la identificación de ataques.

Además, ofrece la posibilidad de entrenar un modelo desarrollado en TensorFlow mediante el uso de aprendizaje federado.

\subsection{Valor del producto}
\label{subsec:Valor}
Este producto es de gran utilidad para el proceso de desarrollo y validación de modelos de inteligencia artificial en un entorno controlado, específico y realista sin la necesidad de invertir recursos en desarrollar una infraestructura costosa y que implique un alto consumo de tiempo.

Se podría decir que es una solución ideal como primer paso en la evaluación de modelos antes de desplegarlos en entornos reales, pudiendo así identificar errores o aspectos a mejorar. También se podría usar como demostración del funcionamiento de un modelo ante diferentes tipos de flujos sobre la red.

\subsection{Público objetivo}
\label{subsec:PublicoObjetivo}
Este proyecto está diseñado para investigadores en ciencia de datos y seguridad informática, así como para personas apasionadas de la inteligencia artificial que buscan un entorno controlado y accesible para probar y desarrollar modelos, sin la necesidad de una infraestructura compleja.

\subsection{Uso previsto}
\label{subsec:UsoPrevisto}
La aplicación está diseñada para ser intuitiva y accesible, permitiendo usarse sin necesidad de registro ni de conocimientos avanzados.

\section{Objetivos generales}
\label{sec:Objetivos}
Este proyecto se ha desarrollado cumpliendo una serie de objetivos básicos:
\begin{itemize}
    \item Implementación de una \textbf{arquitectura federada} para el entrenamiento de modelos desarrollados con TensorFlow.
    \item Desarrollo de un software capaz de \textbf{simular el flujo de red entre dispositivos conectados}, siendo flexible y adaptable a cada necesidad.
    \item \textbf{Detección de ataques} generados dentro del propio entorno de simulación, mediante el uso de modelos de redes neuronales.
    \item Posibilidad para los usuarios de \textbf{crear sus propios ataques} y definir \textbf{flujos} de red personalizados.
    \item Opción de cargar \textbf{modelos propios} para el análisis de trazas de red.
\end{itemize}

\section{Catálogo de requisitos}
\label{sec:CatalogoRequisitos}

\subsection{Requisitos funcionales}
\label{subsec:RequisitosFuncionales}
\begin{itemize}
    \item \textbf{RF-1 Diseño de red}: el usuario debe poder configurar una topología de red de manera sencilla y flexible.
    \begin{itemize}
        \item \textbf{RF-1.1 Añadir nodos}: el usuario debe poder añadir el número indefinido de nodos a la red.
        \item \textbf{RF-1.2 Eliminar nodos}: el usuario debe poder eliminar los nodos de la red.
        \item \textbf{RF-1.3 Modificar nodos}: el usuario debe poder modificar cada una de las propiedades de los nodos según el diseño lo permita.
        \item \textbf{RF-1.4 Conectar nodos}: el usuario debe poder conectar los nodos entre sí.
        \item \textbf{RF-1.5 Modificar conexiones}: el usuario debe poder modificar las propiedades de las conexiones.
        \item \textbf{RF-1.6 Importar configuración}: el usuario debe ser capaz de importar configuraciones guardadas.
        \item \textbf{RF-1.7 Guardar configuración}: el usuario debe ser capaz de guardar topologías de red diseñadas.
    \end{itemize}
    \item \textbf{RF-2 Simulación de flujos}: el usuario debe poder simular flujos de red y que estos estén registrados.
    \begin{itemize}
        \item \textbf{RF-2.1 Ejecutar un comando}: el usuario debe poder ejecutar un comando desde un nodo a uno o varios nodos al mismo tiempo.
        \item \textbf{RF-2.2 Ejecutar un ataque}: el usuario debe poder ejecutar un ataque desde un nodo de tipo ordenador a uno o varios nodos al mismo tiempo.
        \item \textbf{RF-2.3 Interceptar un flujo}: el nodo debe de ser capaz de interceptar un flujo enviado por otro nodo.
        \item \textbf{RF-2.4 Registro de flujos}: cada nodo debe tener un registro de los flujos que han pasado sobre este.
        \item \textbf{RF-2.5 Importar biblioteca}: el usuario debe poder importar una biblioteca de comandos, ataques e interceptores y que aparezcan como disponibles en los nodos.
        \item \textbf{RF-2.6 Eliminar biblioteca}: el usuario debe poder eliminar la biblioteca que ha sido importada.
    \end{itemize}
    \item \textbf{RF-3 Análisis de flujos}: el usuario debe poder analizar los flujos de datos en la red.
    \begin{itemize}
        \item \textbf{RF-3.1 Importar modelo entrenado}: el usuario debe ser capaz de importar uno o varios modelos entrenados con sus correspondientes scripts para su uso en el simulador.
        \item \textbf{RF-3.2 Eliminar modelo entrenado}: el usuario debe ser capaz de eliminar los modelos importados del simulador.
        \item \textbf{RF-3.3 Selección de modelo a usar}: el usuario debe ser capaz de seleccionar en cada conexión que modelo se debe usar para detectar los ataques.
        \item \textbf{RF-3.4 Predicciones de ataques}: el simulador debe ser capaz de mostrar al usuario las predicciones que proporciona el modelo tras pasar las trazas de red.
    \end{itemize}
    \item \textbf{RF-4 Configuración del simulador}: el usuario debe poder configurar todas las opciones de visualización del simulador, así como tener la opción de deshacer y rehacer cambios.    
    \item \textbf{RF-5 Entrenamiento federado}: el usuario debe ser capaz de entrenar un modelo de TensorFlow mediante aprendizaje federado.
    \begin{itemize}
         \item \textbf{RF-5.1 Cargar configuración del modelo base}: el usuario debe ser capaz de cargar la configuración del modelo base a cada cliente y servidor antes del entrenamiento federado.
        \item \textbf{RF-5.2 Dividir dataset}: el usuario debe ser capaz de dividir un dataset en partes iguales para su uso en aprendizaje federado.
        \item \textbf{RF-5.3 Cargar dataset}: el usuario debe ser capaz de cargar los dataset necesarios a cada cliente y servidor antes del entrenamiento federado.
        \item \textbf{RF-5.4 Servidor federado}: el usuario debe ser capaz de lanzar un servidor federado con posibles configuraciones para coordinar el entrenamiento distribuido.
        \begin{itemize}
            \item \textbf{RF-5.4.1 Entrenamiento modelo global}: el servidor debe ser capaz de organizar el entrenamiento global del modelo.
            \item \textbf{RF-5.4.2 Evaluación modelo global}: el servidor debe ser capaz de organizar la evaluación global del modelo.
            \item \textbf{RF-5.4.3 Exportar modelo global}: el servidor debe ser capaz de exportar el modelo global al finalizar el entrenamiento.
        \end{itemize}
        \item \textbf{RF-5.5 Cliente federado}: el usuario debe ser capaz de lanzar varios clientes federados con posibles configuraciones desde distintos equipos y conectarlos al servidor.
        \begin{itemize}
            \item \textbf{RF-5.5.1 Entrenamiento modelo}: los clientes deben ser capaces de entrenar el modelo global.
            \item \textbf{RF-5.5.2 Evaluación modelo}: los clientes deben ser capaces de evaluar el modelo global.
        \end{itemize}   
        \item \textbf{RF-5.6 Métricas}: el usuario debe ser capaz de visualizar las métricas del entrenamiento y la evaluación.
    \end{itemize}
\end{itemize}

\subsection{Requisitos no funcionales}
\label{subsec:RequisitosNoFuncionales}
\begin{itemize}
    \item \textbf{RNF-1 Usabilidad}: la interfaz debe de ser intuitiva y accesible para usuarios no expertos, además de no requerir configuraciones complejas para su uso.
    \item \textbf{RNF-2 Rendimiento y confiabilidad}: las simulaciones deben de ejecutarse en tiempos razonables. También es esperable que se gestionen correctamente los errores y que en caso de actualización de la página no se pierda el estado de la aplicación.
    \item \textbf{RNF-3 Compatibilidad}: debe de ser compatible con la mayoría de los navegadores modernos y no requerir de funciones especiales que sean necesaria la configuración.
    \item \textbf{RNF-4 Seguridad}: debe de ser un software estable ante posibles fallos que cometa el usuario, además de no comprometer al sistema.
    \item \textbf{RNF-5 Mantenibilidad y escalabilidad}: el código debe de estar bien estructurado y la documentación debe ser clara y concisa.
\end{itemize}

\section{Especificación de requisitos}
\label{sec:EspecificacionRequisitos}

\imagen{casos_de_uso}{Diagrama de casos de uso}{1.1}
\newpage

\begin{table}[p]
	\centering
	\begin{tabularx}{\linewidth}{ p{0.21\columnwidth} p{0.71\columnwidth} }
		\toprule
		\textbf{CU-1}    & \textbf{Diseño de red}\\
		\toprule
		\textbf{Versión}              & 1.0    \\
		\textbf{Autor}                & Alejandro Diez Bermejo \\
		\textbf{Requisitos asociados} & RF-1, RF-1.1, RF-1.2, RF-1.3, RF-1.4, RF-1.5, RF-1.6, RF-1.7 \\
		\textbf{Descripción}          & El usuario quiere modificar la topología de la red \\
		\textbf{Precondición}         & El usuario tiene conexión a internet \\
                                      & El simulador está disponible \\
		\textbf{Acciones}             &
		\begin{enumerate}\def\labelenumi{\arabic{enumi}.}\tightlist
			\item El usuario entra al simulador
			\item El simulador muestra la interfaz
            \item El simulador inicia un proyecto nuevo
		\end{enumerate}\\
		\textbf{Postcondición}        & Proyecto nuevo \\
		\textbf{Excepciones}          & - \\
		\textbf{Importancia}          & Alta \\
		\bottomrule
	\end{tabularx}
	\caption{CU-1 Diseño de red}
\end{table}

\begin{table}[p]
	\centering
	\begin{tabularx}{\linewidth}{ p{0.21\columnwidth} p{0.71\columnwidth} }
		\toprule
		\textbf{CU-2}    & \textbf{Añadir nodos a la red}\\
		\toprule
		\textbf{Versión}              & 1.0    \\
		\textbf{Autor}                & Alejandro Diez Bermejo \\
		\textbf{Requisitos asociados} & RF-1.1 \\
		\textbf{Descripción}          & El usuario añade uno o varios nodos a la red mediante la interfaz gráfica \\
		\textbf{Precondición}         & El usuario está dentro del simulador \\
		\textbf{Acciones}             &
		\begin{enumerate}\def\labelenumi{\arabic{enumi}.}\tightlist
			\item El usuario selecciona la opción ``Añadir router'' (en caso de añadir un router) o ``Añadir dispositivo'' (en caso de añadir un ordenador o dispositivo IoT)
			\item El simulador pide al usuario datos del nodo (nombre y tipo)
            \item El usuario completa el formulario
            \item El simulador añade el nodo a la red
		\end{enumerate}\\
		\textbf{Postcondición}        & El nodo se ha añadido a la red \\
		\textbf{Excepciones}          & Falta algún dato de rellenar en el formulario \\
                                      & Existe un nodo de tipo router en la red \\
		\textbf{Importancia}          & Alta \\
		\bottomrule
	\end{tabularx}
	\caption{CU-2 Añadir nodos a la red}
\end{table}

\begin{table}[p]
	\centering
	\begin{tabularx}{\linewidth}{ p{0.21\columnwidth} p{0.71\columnwidth} }
		\toprule
		\textbf{CU-3}    & \textbf{Eliminar nodos de la red}\\
		\toprule
		\textbf{Versión}              & 1.0    \\
		\textbf{Autor}                & Alejandro Diez Bermejo \\
		\textbf{Requisitos asociados} & RF-1.2 \\
		\textbf{Descripción}          & El usuario elimina un nodo de la red mediante la interfaz gráfica \\
		\textbf{Precondición}         & El usuario está dentro del simulador \\
        		                      & Existe el nodo a eliminar \\
		\textbf{Acciones}             &
		\begin{enumerate}
			\def\labelenumi{\arabic{enumi}.}
			\tightlist
			\item El usuario selecciona un nodo a eliminar
			\item El simulador solicita la confirmación de eliminación
            \item El simulador desconecta el nodo y lo elimina
		\end{enumerate}\\
		\textbf{Postcondición}        & El nodo y las conexiones de este se eliminan \\
		\textbf{Excepciones}          & - \\
		\textbf{Importancia}          & Media \\
		\bottomrule
	\end{tabularx}
	\caption{CU-3 Eliminar nodos de la red}
\end{table}

\begin{table}[p]
	\centering
	\begin{tabularx}{\linewidth}{ p{0.21\columnwidth} p{0.71\columnwidth} }
		\toprule
		\textbf{CU-4}    & \textbf{Modificar propiedades del nodo}\\
		\toprule
		\textbf{Versión}              & 1.0    \\
		\textbf{Autor}                & Alejandro Diez Bermejo \\
		\textbf{Requisitos asociados} & RF-1.3 \\
		\textbf{Descripción}          & El usuario modifica las propiedades de un nodo \\
		\textbf{Precondición}         & El usuario está dentro del simulador \\
                                      & Existe el nodo a modificar \\
		\textbf{Acciones}             &
		\begin{enumerate}
			\def\labelenumi{\arabic{enumi}.}
			\tightlist
			\item El usuario selecciona un nodo a modificar
            \item El simulador muestra las propiedades del nodo
            \item El usuario modifica las propiedades
            \item El simulador guarda los cambios
		\end{enumerate}\\
		\textbf{Postcondición}        & El nodo guarda las modificaciones \\
		\textbf{Excepciones}          & Valores no válidos \\
		\textbf{Importancia}          & Baja \\
		\bottomrule
	\end{tabularx}
	\caption{CU-4 Modificar propiedades del nodo}
\end{table}

\begin{table}[p]
	\centering
	\begin{tabularx}{\linewidth}{ p{0.21\columnwidth} p{0.71\columnwidth} }
		\toprule
		\textbf{CU-5}    & \textbf{Conectar los nodos}\\
		\toprule
		\textbf{Versión}              & 1.0    \\
		\textbf{Autor}                & Alejandro Diez Bermejo \\
		\textbf{Requisitos asociados} & RF-1.4 \\
		\textbf{Descripción}          & El usuario conecta dos nodos \\
		\textbf{Precondición}         & El usuario está dentro del simulador \\
                                      & Existe un dispositivo sin conectar \\
                                      & Existe un router al cual conectarse \\
		\textbf{Acciones}             &
		\begin{enumerate}
			\def\labelenumi{\arabic{enumi}.}
			\tightlist
			\item El usuario selecciona un dispositivo a conectar
            \item El simulador muestra la opción de conectarse al router
            \item El usuario selecciona la opción de conectarse
            \item El simulador conecta los nodos y asigna una IP al dispositivo
		\end{enumerate}\\
		\textbf{Postcondición}        & El nodo está conectado y tiene una IP asignada \\
		\textbf{Excepciones}          & - \\
		\textbf{Importancia}          & Alta \\
		\bottomrule
	\end{tabularx}
	\caption{CU-5 Conectar los nodos}
\end{table}

\begin{table}[p]
	\centering
	\begin{tabularx}{\linewidth}{ p{0.21\columnwidth} p{0.71\columnwidth} }
		\toprule
		\textbf{CU-6}    & \textbf{Modificar propiedades de la conexión}\\
		\toprule
		\textbf{Versión}              & 1.0    \\
		\textbf{Autor}                & Alejandro Diez Bermejo \\
		\textbf{Requisitos asociados} & RF-1.5 \\
		\textbf{Descripción}          & El usuario modifica las propiedades de una conexión \\
		\textbf{Precondición}         & El usuario está dentro del simulador \\
                                      & Existe una conexión a modificar \\
		\textbf{Acciones}             &
		\begin{enumerate}
			\def\labelenumi{\arabic{enumi}.}
			\tightlist
			\item El usuario selecciona la conexión a modificar
            \item El simulador muestra las propiedades de la conexión
            \item El usuario modifica las propiedades
            \item El simulador guarda los cambios
		\end{enumerate}\\
		\textbf{Postcondición}        & La conexión guarda las modificaciones \\
		\textbf{Excepciones}          & Valores no válidos \\
		\textbf{Importancia}          & Media \\
		\bottomrule
	\end{tabularx}
	\caption{CU-6 Modificar propiedades de la conexión}
\end{table}

\begin{table}[p]
	\centering
	\begin{tabularx}{\linewidth}{ p{0.21\columnwidth} p{0.71\columnwidth} }
		\toprule
		\textbf{CU-7}    & \textbf{Importación configuración de red}\\
		\toprule
		\textbf{Versión}              & 1.0    \\
		\textbf{Autor}                & Alejandro Diez Bermejo \\
		\textbf{Requisitos asociados} & RF-1.6 \\
		\textbf{Descripción}          & El usuario importa una configuración guardada \\
        \textbf{Precondición}         & El usuario está dentro del simulador \\
        		                      & El usuario dispone de un archivo de configuración válido \\
		\textbf{Acciones}             &
		\begin{enumerate}
			\def\labelenumi{\arabic{enumi}.}
			\tightlist
			\item El usuario selecciona la opción de ``Abrir''
            \item Elige un archivo
            \item El simulador carga la configuración
		\end{enumerate}\\
		\textbf{Postcondición}        & La red se muestra como estaba guardada \\
		\textbf{Excepciones}          & Archivo incompatible \\
		\textbf{Importancia}          & Baja \\
		\bottomrule
	\end{tabularx}
	\caption{CU-7 Importación configuración de red}
\end{table}

\begin{table}[p]
	\centering
	\begin{tabularx}{\linewidth}{ p{0.21\columnwidth} p{0.71\columnwidth} }
		\toprule
		\textbf{CU-8}    & \textbf{Guardar configuración de red}\\
		\toprule
		\textbf{Versión}              & 1.0    \\
		\textbf{Autor}                & Alejandro Diez Bermejo \\
		\textbf{Requisitos asociados} & RF-1.7 \\
		\textbf{Descripción}          & El usuario guarda la configuración de red actual \\
        \textbf{Precondición}         & El usuario está dentro del simulador \\
		\textbf{Acciones}             &
		\begin{enumerate}
			\def\labelenumi{\arabic{enumi}.}
			\tightlist
			\item El usuario selecciona la opción de ``Guardar''
            \item El simulador exporta la configuración de la red actual
		\end{enumerate}\\
		\textbf{Postcondición}        & El archivo se descarga \\
		\textbf{Excepciones}          & - \\
		\textbf{Importancia}          & Baja \\
		\bottomrule
	\end{tabularx}
	\caption{CU-8 Guardar configuración de red}
\end{table}

\begin{table}[p]
	\centering
	\begin{tabularx}{\linewidth}{ p{0.21\columnwidth} p{0.71\columnwidth} }
		\toprule
		\textbf{CU-9}    & \textbf{Simulación de flujos}\\
		\toprule
		\textbf{Versión}              & 1.0    \\
		\textbf{Autor}                & Alejandro Diez Bermejo \\
		\textbf{Requisitos asociados} & RF-2, RF-2.1, RF-2.2, RF-2.3, RF-2.4, RF-2.5, RF-2.6 \\
		\textbf{Descripción}          & El usuario quiere realizar simulaciones de flujos de red \\
        \textbf{Precondición}         & El usuario está dentro del simulador \\
                                      & Hay una red diseñada \\
                                      & Hay nodos conectados \\
		\textbf{Acciones}             &
		\begin{enumerate}
			\def\labelenumi{\arabic{enumi}.}
			\tightlist
			\item El usuario entra al simulador
            \item El simulador muestra la interfaz
            \item El simulador muestra la configuración de red actual
		\end{enumerate}\\
		\textbf{Postcondición}        & Configuración de red actual \\
		\textbf{Excepciones}          & - \\
		\textbf{Importancia}          & Alta \\
		\bottomrule
	\end{tabularx}
	\caption{CU-9 Simulación de flujos}
\end{table}

\begin{table}[p]
	\centering
	\begin{tabularx}{\linewidth}{ p{0.21\columnwidth} p{0.71\columnwidth} }
		\toprule
		\textbf{CU-10}    & \textbf{Ejecutar un comando}\\
		\toprule
		\textbf{Versión}              & 1.0    \\
		\textbf{Autor}                & Alejandro Diez Bermejo \\
		\textbf{Requisitos asociados} & RF-2.1 \\
		\textbf{Descripción}          & El usuario quiere ejecutar un comando desde un nodo hacia otros \\
        \textbf{Precondición}         & El usuario está dentro del simulador \\
                                      & Hay nodos conectados \\
                                      & El simulador dispone de comandos \\
		\textbf{Acciones}             &
		\begin{enumerate}
			\def\labelenumi{\arabic{enumi}.}
			\tightlist
			\item El usuario selecciona el nodo origen
            \item El simulador muestra la vista de tráfico de red
            \item El usuario selecciona el comando a enviar
            \item El usuario selecciona los objetivos
            \item El usuario selecciona la opción de enviar
            \item El simulador simula el flujo
		\end{enumerate}\\
		\textbf{Postcondición}        & Flujo registrado en los nodos \\
		\textbf{Excepciones}          & Valores no válidos \\
		\textbf{Importancia}          & Alta \\
		\bottomrule
	\end{tabularx}
	\caption{CU-10 Ejecutar un comando}
\end{table}

\begin{table}[p]
	\centering
	\begin{tabularx}{\linewidth}{ p{0.21\columnwidth} p{0.71\columnwidth} }
		\toprule
		\textbf{CU-11}    & \textbf{Lanzar un ataque}\\
		\toprule
		\textbf{Versión}              & 1.0    \\
		\textbf{Autor}                & Alejandro Diez Bermejo \\
		\textbf{Requisitos asociados} & RF-2.2 \\
		\textbf{Descripción}          & El usuario quiere lanzar un ataque desde un nodo hacia otros \\
        \textbf{Precondición}         & El usuario está dentro del simulador \\
                                      & Hay nodos conectados \\
                                      & Existe un ordenador conectado \\
                                      & El ordenador dispone de ataques \\
		\textbf{Acciones}             &
		\begin{enumerate}
			\def\labelenumi{\arabic{enumi}.}
			\tightlist
			\item El usuario selecciona el ordenador atacante
            \item El simulador muestra la vista de tráfico de red
            \item El usuario selecciona la sección de ataques
            \item El usuario selecciona el ataque a enviar
            \item El usuario selecciona los objetivos
            \item El usuario selecciona la opción de atacar
            \item El simulador simula el ataque
		\end{enumerate}\\
		\textbf{Postcondición}        & Flujo registrado en los nodos \\
		\textbf{Excepciones}          & Valores no válidos \\
		\textbf{Importancia}          & Alta \\
		\bottomrule
	\end{tabularx}
	\caption{CU-11 Lanzar un ataque}
\end{table}

\begin{table}[p]
	\centering
	\begin{tabularx}{\linewidth}{ p{0.21\columnwidth} p{0.71\columnwidth} }
		\toprule
		\textbf{CU-12}    & \textbf{Interceptar flujos}\\
		\toprule
		\textbf{Versión}              & 1.0    \\
		\textbf{Autor}                & Alejandro Diez Bermejo \\
		\textbf{Requisitos asociados} & RF-2.3 \\
		\textbf{Descripción}          & Un nodo puede capturar los flujos que pasan por el \\
        \textbf{Precondición}         & Existe un nodo conectado \\
		\textbf{Acciones}             &
		\begin{enumerate}
			\def\labelenumi{\arabic{enumi}.}
			\tightlist
			\item El flujo llega al nodo
            \item El nodo intercepta el tráfico
            \item El nodo realiza una acción
        \end{enumerate}\\
		\textbf{Postcondición}        & Acción respuesta \\
		\textbf{Excepciones}          & - \\
		\textbf{Importancia}          & Alta \\
		\bottomrule
	\end{tabularx}
	\caption{CU-12 Interceptar flujos}
\end{table}

\begin{table}[p]
	\centering
	\begin{tabularx}{\linewidth}{ p{0.21\columnwidth} p{0.71\columnwidth} }
		\toprule
		\textbf{CU-13}    & \textbf{Registros de flujos}\\
		\toprule
		\textbf{Versión}              & 1.0    \\
		\textbf{Autor}                & Alejandro Diez Bermejo \\
		\textbf{Requisitos asociados} & RF-2.4 \\
		\textbf{Descripción}          & Un nodo mantiene un registro de los flujos que pasan por el \\
        \textbf{Precondición}         & Se ha iniciado al menos un flujo en la red \\
		\textbf{Acciones}             &
		\begin{enumerate}
			\def\labelenumi{\arabic{enumi}.}
			\tightlist
            \item El flujo llega al nodo
            \item El nodo registra el flujo en el historial
		\end{enumerate}\\
		\textbf{Postcondición}        & Historial de flujos actualizado \\
		\textbf{Excepciones}          & - \\
		\textbf{Importancia}          & Alta \\
		\bottomrule
	\end{tabularx}
	\caption{CU-13 Registros de flujos}
\end{table}

\begin{table}[p]
	\centering
	\begin{tabularx}{\linewidth}{ p{0.21\columnwidth} p{0.71\columnwidth} }
		\toprule
		\textbf{CU-14}    & \textbf{Importar biblioteca de comandos, ataques e interceptores}\\
		\toprule
		\textbf{Versión}              & 1.0    \\
		\textbf{Autor}                & Alejandro Diez Bermejo \\
		\textbf{Requisitos asociados} & RF-2.5 \\
		\textbf{Descripción}          & El usuario importa una biblioteca externa de scripts \\
        \textbf{Precondición}         & El usuario está dentro del simulador \\
                                      & Archivo válido \\
		\textbf{Acciones}             &
		\begin{enumerate}
			\def\labelenumi{\arabic{enumi}.}
			\tightlist
			\item El usuario selecciona ``Importar biblioteca externa''
            \item El usuario selecciona un archivo a cargar
            \item El simulador valida e importa las posibles funciones de la biblioteca
		\end{enumerate}\\
		\textbf{Postcondición}        & Comandos, ataques e interceptores disponibles \\
		\textbf{Excepciones}          & Archivo no válido o con nombres de funciones incorrectas \\
		\textbf{Importancia}          & Media \\
		\bottomrule
	\end{tabularx}
	\caption{CU-14 Importar biblioteca de comandos, ataques e interceptores}
\end{table}

\begin{table}[p]
	\centering
	\begin{tabularx}{\linewidth}{ p{0.21\columnwidth} p{0.71\columnwidth} }
		\toprule
		\textbf{CU-15}    & \textbf{Eliminar biblioteca importada}\\
		\toprule
		\textbf{Versión}              & 1.0    \\
		\textbf{Autor}                & Alejandro Diez Bermejo \\
		\textbf{Requisitos asociados} & RF-2.6 \\
		\textbf{Descripción}          & El usuario elimina una biblioteca que había sido cargada \\
        \textbf{Precondición}         & El usuario está dentro del simulador \\
                                      & El simulador tiene una biblioteca cargada \\
		\textbf{Acciones}             &
		\begin{enumerate}
			\def\labelenumi{\arabic{enumi}.}
			\tightlist
			\item El usuario selecciona ``Eliminar biblioteca externa''
            \item El simulador elimina la biblioteca de todos los nodos
		\end{enumerate}\\
		\textbf{Postcondición}        & Biblioteca externa eliminada \\
		\textbf{Excepciones}          & - \\
		\textbf{Importancia}          & Media \\
		\bottomrule
	\end{tabularx}
	\caption{CU-15 Eliminar biblioteca importada}
\end{table}

\begin{table}[p]
	\centering
	\begin{tabularx}{\linewidth}{ p{0.21\columnwidth} p{0.71\columnwidth} }
		\toprule
		\textbf{CU-16}    & \textbf{Análisis de flujos en la red}\\
		\toprule
		\textbf{Versión}              & 1.0    \\
		\textbf{Autor}                & Alejandro Diez Bermejo \\
		\textbf{Requisitos asociados} & RF-3, RF-3.1, RF-3.2, RF-3.3, RF-3.4 \\
		\textbf{Descripción}          & El usuario quiere analizar los flujos de datos en la red \\
        \textbf{Precondición}         & El usuario está dentro del simulador \\
                                      & Hay una topología de red establecida \\
		\textbf{Acciones}             &
		\begin{enumerate}
			\def\labelenumi{\arabic{enumi}.}
			\tightlist
			\item El usuario entra al simulador
            \item El simulador muestra la interfaz
            \item El simulador muestra la configuración de red actual
		\end{enumerate}\\
		\textbf{Postcondición}        & Configuración de red actual \\
		\textbf{Excepciones}          & - \\
		\textbf{Importancia}          & Alta \\
		\bottomrule
	\end{tabularx}
	\caption{CU-16 Análisis de flujos en la red}
\end{table}

\begin{table}[p]
	\centering
	\begin{tabularx}{\linewidth}{ p{0.21\columnwidth} p{0.71\columnwidth} }
		\toprule
		\textbf{CU-17}    & \textbf{Importar modelo entrenado}\\
		\toprule
		\textbf{Versión}              & 1.0    \\
		\textbf{Autor}                & Alejandro Diez Bermejo \\
		\textbf{Requisitos asociados} & RF-3.1 \\
		\textbf{Descripción}          & El usuario importa modelos de detección de ataques \\
        \textbf{Precondición}         & El usuario está dentro del simulador \\
                                      & El usuario dispone de modelos entrenados con su script de análisis correspondiente \\
		\textbf{Acciones}             &
		\begin{enumerate}
			\def\labelenumi{\arabic{enumi}.}
			\tightlist
			\item El usuario selecciona ``Importar modelos''
            \item El usuario selecciona los modelos a cargar
            \item El simulador valida e importa los modelos
		\end{enumerate}\\
		\textbf{Postcondición}        & Listas de modelos a utilizar \\
		\textbf{Excepciones}          & Modelos no válidos o un script que no sigue la estructura adecuada \\
		\textbf{Importancia}          & Alta \\
		\bottomrule
	\end{tabularx}
	\caption{CU-17 Importar modelo entrenado}
\end{table}

\begin{table}[p]
	\centering
	\begin{tabularx}{\linewidth}{ p{0.21\columnwidth} p{0.71\columnwidth} }
		\toprule
		\textbf{CU-18}    & \textbf{Eliminar modelos importados}\\
		\toprule
		\textbf{Versión}              & 1.0    \\
		\textbf{Autor}                & Alejandro Diez Bermejo \\
		\textbf{Requisitos asociados} & RF-3.2 \\
		\textbf{Descripción}          & El usuario elimina los modelos que habían sido cargados \\
        \textbf{Precondición}         & El usuario está dentro del simulador \\
                                      & El simulador tiene modelos cargados \\
		\textbf{Acciones}             &
		\begin{enumerate}
			\def\labelenumi{\arabic{enumi}.}
			\tightlist
			\item El usuario selecciona ``Eliminar modelos''
            \item El simulador elimina todos los modelos de las conexiones
		\end{enumerate}\\
		\textbf{Postcondición}        & Modelos eliminados \\
		\textbf{Excepciones}          & - \\
		\textbf{Importancia}          & Media \\
		\bottomrule
	\end{tabularx}
	\caption{CU-18 Eliminar modelos importados}
\end{table}

\begin{table}[p]
	\centering
	\begin{tabularx}{\linewidth}{ p{0.21\columnwidth} p{0.71\columnwidth} }
		\toprule
		\textbf{CU-19}    & \textbf{Seleccionar modelo de detección}\\
		\toprule
		\textbf{Versión}              & 1.0    \\
		\textbf{Autor}                & Alejandro Diez Bermejo \\
		\textbf{Requisitos asociados} & RF-3.3 \\
		\textbf{Descripción}          & El usuario selecciona qué modelo usar en cada conexión para la detección de ataques \\
        \textbf{Precondición}         & El usuario está dentro del simulador \\
                                      & El simulador tiene modelos cargados \\
                                      & El simulador tiene nodos conectados \\
		\textbf{Acciones}             &
		\begin{enumerate}
			\def\labelenumi{\arabic{enumi}.}
			\tightlist
			\item El usuario accede a la configuración de una conexión
            \item El usuario selecciona un modelo disponible
            \item El simulador guarda los cambios
		\end{enumerate}\\
		\textbf{Postcondición}        & Modelos eliminados \\
		\textbf{Excepciones}          & - \\
		\textbf{Importancia}          & Media \\
		\bottomrule
	\end{tabularx}
	\caption{CU-19 Seleccionar modelo de detección}
\end{table}

\begin{table}[p]
	\centering
	\begin{tabularx}{\linewidth}{ p{0.21\columnwidth} p{0.71\columnwidth} }
		\toprule
		\textbf{CU-20}    & \textbf{Ver predicciones de ataques}\\
		\toprule
		\textbf{Versión}              & 1.0    \\
		\textbf{Autor}                & Alejandro Diez Bermejo \\
		\textbf{Requisitos asociados} & RF-3.4 \\
		\textbf{Descripción}          & El simulador muestra predicciones sobre los flujos interceptados \\
        \textbf{Precondición}         & El usuario está dentro del simulador \\
                                      & Existe nodos conectados \\
                                      & La conexión tiene un modelo activo \\
		\textbf{Acciones}             &
		\begin{enumerate}
			\def\labelenumi{\arabic{enumi}.}
			\tightlist
			\item Un paquete pasa por una conexión
            \item El simulador llama a la función ``analizar'' del script que fue importada con el modelo
            \item El simulador cambia de color la conexión según el resultado
		\end{enumerate}\\
		\textbf{Postcondición}        & Resultado de la predicción \\
		\textbf{Excepciones}          & Fallo en el uso del script \\
		\textbf{Importancia}          & Alta \\
		\bottomrule
	\end{tabularx}
	\caption{CU-20 Ver predicciones de ataques}
\end{table}

\begin{table}[p]
	\centering
	\begin{tabularx}{\linewidth}{ p{0.21\columnwidth} p{0.71\columnwidth} }
		\toprule
		\textbf{CU-21}    & \textbf{Configurar el simulador}\\
		\toprule
		\textbf{Versión}              & 1.0    \\
		\textbf{Autor}                & Alejandro Diez Bermejo \\
		\textbf{Requisitos asociados} & RF-4 \\
		\textbf{Descripción}          & El usuario quiere modificar las opciones de visualización y gestionar los estados de esta \\
        \textbf{Precondición}         & El usuario está dentro del simulador \\
		\textbf{Acciones}             &
		\begin{enumerate}
			\def\labelenumi{\arabic{enumi}.}
			\tightlist
			\item El usuario busca en el menú la configuración a modificar
            \item El usuario modifica los cambios que desea
            \item El simulador muestra los nuevos cambios
		\end{enumerate}\\
		\textbf{Postcondición}        & Nuevos cambios aplicados \\
		\textbf{Excepciones}          & - \\
		\textbf{Importancia}          & Baja \\
		\bottomrule
	\end{tabularx}
	\caption{CU-21 Configurar el simulador}
\end{table}

\begin{table}[p]
	\centering
	\begin{tabularx}{\linewidth}{ p{0.21\columnwidth} p{0.71\columnwidth} }
		\toprule
		\textbf{CU-22}    & \textbf{Ejecutar entrenamiento federado}\\
		\toprule
		\textbf{Versión}              & 1.0    \\
		\textbf{Autor}                & Alejandro Diez Bermejo \\
		\textbf{Requisitos asociados} & RF-5, RF-5.1, RF-5.2, RF-5.3, RF-5.4, RF-5.4.1, RF-5.4.2, RF-5.4.3, RF-5.5, RF-5.5.1, RF-5.5.2, RF-5.6 \\
		\textbf{Descripción}          & El usuario quiere entrena un modelo de redes neuronales de forma federada \\
        \textbf{Precondición}         & Modelo base \\
                                      & Datasets \\
                                      & Tiene instaladas todas las dependencias \\
		\textbf{Acciones}             &
		\begin{enumerate}
			\def\labelenumi{\arabic{enumi}.}
			\tightlist
            \item El usuario accede al módulo de entrenamiento federado
            \item El usuario ejecuta el script
            \item El script muestra las posibles ejecuciones disponibles
		\end{enumerate}\\
		\textbf{Postcondición}        & Listado de posibles ejecuciones \\
		\textbf{Excepciones}          & - \\
		\textbf{Importancia}          & Alta \\
		\bottomrule
	\end{tabularx}
	\caption{CU-22 Ejecutar entrenamiento federado}
\end{table}

\begin{table}[p]
	\centering
	\begin{tabularx}{\linewidth}{ p{0.21\columnwidth} p{0.71\columnwidth} }
		\toprule
		\textbf{CU-23}    & \textbf{Cargar configuración del modelo base}\\
		\toprule
		\textbf{Versión}              & 1.0    \\
		\textbf{Autor}                & Alejandro Diez Bermejo \\
		\textbf{Requisitos asociados} & RF-5.1 \\
		\textbf{Descripción}          & El usuario quiere cargar la configuración base del modelo \\
        \textbf{Precondición}         & Archivo base del modelo \\
		\textbf{Acciones}             &
		\begin{enumerate}
			\def\labelenumi{\arabic{enumi}.}
			\tightlist
            \item El usuario ejecuta el script con el argumento \texttt{-{}-model} y la ruta del modelo
            \item El script carga el modelo en memoria
		\end{enumerate}\\
		\textbf{Postcondición}        & Modelo cargado en memoria \\
		\textbf{Excepciones}          & No se ha encontrado el modelo \\
                                      & Errores en la configuración del modelo \\
		\textbf{Importancia}          & Alta \\
		\bottomrule
	\end{tabularx}
	\caption{CU-23 Cargar configuración del modelo base}
\end{table}

\begin{table}[p]
	\centering
	\begin{tabularx}{\linewidth}{ p{0.21\columnwidth} p{0.71\columnwidth} }
		\toprule
		\textbf{CU-24}    & \textbf{Dividir dataset para entrenamiento federado}\\
		\toprule
		\textbf{Versión}              & 1.0    \\
		\textbf{Autor}                & Alejandro Diez Bermejo \\
		\textbf{Requisitos asociados} & RF-5.2 \\
		\textbf{Descripción}          & El usuario quiere dividir un dataset en partes iguales para los clientes federados \\
        \textbf{Precondición}         & Dataset está cargado \\
		\textbf{Acciones}             &
		\begin{enumerate}
			\def\labelenumi{\arabic{enumi}.}
			\tightlist
            \item El usuario ejecuta el script con el argumento \texttt{-d} seguido del número de archivos a crear
            \item El script genera los archivos
		\end{enumerate}\\
		\textbf{Postcondición}        & Archivos divididos \\
		\textbf{Excepciones}          & - \\
		\textbf{Importancia}          & Media \\
		\bottomrule
	\end{tabularx}
	\caption{CU-24 Dividir dataset para entrenamiento federado}
\end{table}

\begin{table}[p]
	\centering
	\begin{tabularx}{\linewidth}{ p{0.21\columnwidth} p{0.71\columnwidth} }
		\toprule
		\textbf{CU-25}    & \textbf{Cargar dataset}\\
		\toprule
		\textbf{Versión}              & 1.0    \\
		\textbf{Autor}                & Alejandro Diez Bermejo \\
		\textbf{Requisitos asociados} & RF-5.3 \\
		\textbf{Descripción}          & El usuario quiere cargar el dataset en el cliente o servidor \\
        \textbf{Precondición}         & Dataset válido \\
		\textbf{Acciones}             &
		\begin{enumerate}
			\def\labelenumi{\arabic{enumi}.}
			\tightlist
            \item El usuario ejecuta el script con el argumento \texttt{-{}-train} (entrenamiento) o \texttt{-{}-test} (validación) seguido de la ruta de los archivos de datos y etiquetas
            \item El script carga en memoria los datasets
		\end{enumerate}\\
		\textbf{Postcondición}        & Datasets cargados en memoria \\
		\textbf{Excepciones}          & No se han encontrado los datasets \\
                                      & Error en los archivos \\
		\textbf{Importancia}          &  \\
		\bottomrule
	\end{tabularx}
	\caption{CU-25 Cargar dataset}
\end{table}

\begin{table}[p]
	\centering
	\begin{tabularx}{\linewidth}{ p{0.21\columnwidth} p{0.71\columnwidth} }
		\toprule
		\textbf{CU-26}    & \textbf{Iniciar servidor federado}\\
		\toprule
		\textbf{Versión}              & 1.0    \\
		\textbf{Autor}                & Alejandro Diez Bermejo \\
		\textbf{Requisitos asociados} & RF-5.4, RF-5.4.1, RF-5.4.2, RF-5.4.3 \\
		\textbf{Descripción}          & El usuario quiere iniciar un servidor de aprendizaje federado con parámetros configurables \\
        \textbf{Precondición}         & Hay acceso a internet \\
                                      & Modelo base está cargado \\
                                      & Datasets están cargados \\
		\textbf{Acciones}             &
		\begin{enumerate}
			\def\labelenumi{\arabic{enumi}.}
			\tightlist
            \item El usuario ejecuta el script con el argumento \texttt{-s} y la dirección IP del servidor junto a su puerto
            \item El usuario indica configuraciones adicionales como \texttt{-{}-batch-size} o \texttt{-{}-rounds}
            \item El script inicializa el servidor
            \item El servidor está pendiente de posibles clientes
		\end{enumerate}\\
		\textbf{Postcondición}        & Servidor federado inicializado \\
		\textbf{Excepciones}          & IP o puerto ocupado \\
		\textbf{Importancia}          & Alta \\
		\bottomrule
	\end{tabularx}
	\caption{CU-26 Iniciar servidor federado}
\end{table}

\begin{table}[p]
	\centering
	\begin{tabularx}{\linewidth}{ p{0.21\columnwidth} p{0.71\columnwidth} }
		\toprule
		\textbf{CU-27}    & \textbf{Entrenamiento del modelo global}\\
		\toprule
		\textbf{Versión}              & 1.0    \\
		\textbf{Autor}                & Alejandro Diez Bermejo \\
		\textbf{Requisitos asociados} & RF-5.4.1 \\
		\textbf{Descripción}          & El servidor organiza un entrenamiento global del modelo \\
        \textbf{Precondición}         & Existen clientes conectados \\
		\textbf{Acciones}             &
		\begin{enumerate}
			\def\labelenumi{\arabic{enumi}.}
			\tightlist
            \item El servidor elige diferentes clientes para que entrenen el modelo
            \item El servidor recibe los modelos entrenados por los clientes
            \item El servidor aplica un algoritmo de agregación 
            \item Se actualiza el modelo global
		\end{enumerate}\\
		\textbf{Postcondición}        & Nuevo modelo global \\
		\textbf{Excepciones}          & Fallo en el entrenamiento de clientes \\
                                      & Modelos de clientes diferentes \\
		\textbf{Importancia}          & Alta \\
		\bottomrule
	\end{tabularx}
	\caption{CU-27 Entrenamiento del modelo global}
\end{table}

\begin{table}[p]
	\centering
	\begin{tabularx}{\linewidth}{ p{0.21\columnwidth} p{0.71\columnwidth} }
		\toprule
		\textbf{CU-28}    & \textbf{Evaluar modelo global}\\
		\toprule
		\textbf{Versión}              & 1.0    \\
		\textbf{Autor}                & Alejandro Diez Bermejo \\
		\textbf{Requisitos asociados} & RF-5.4.2 \\
		\textbf{Descripción}          & El servidor organiza una evaluación global del modelo \\
        \textbf{Precondición}         & Dataset de validación \\
                                      & Clientes conectados \\
		\textbf{Acciones}             &
		\begin{enumerate}
			\def\labelenumi{\arabic{enumi}.}
			\tightlist
            \item El servidor elige diferentes clientes para que evalúen el modelo
            \item El servidor evalúa el modelo
            \item El servidor recibe las métricas de evaluación de los clientes
            \item El servidor guarda las métricas de evaluación
		\end{enumerate}\\
		\textbf{Postcondición}        & Métricas de evaluación \\
		\textbf{Excepciones}          & Fallo en la evaluación de los clientes \\
		\textbf{Importancia}          & Alta \\
		\bottomrule
	\end{tabularx}
	\caption{CU-28 Evaluar modelo global}
\end{table}

\begin{table}[p]
	\centering
	\begin{tabularx}{\linewidth}{ p{0.21\columnwidth} p{0.71\columnwidth} }
		\toprule
		\textbf{CU-29}    & \textbf{Exportar modelo global}\\
		\toprule
		\textbf{Versión}              & 1.0    \\
		\textbf{Autor}                & Alejandro Diez Bermejo \\
		\textbf{Requisitos asociados} & RF-5.4.3 \\
		\textbf{Descripción}          & Una vez terminado el entrenamiento, el servidor exporta el modelo global resultante \\
        \textbf{Precondición}         & Se ha terminado el entrenamiento del modelo \\
		\textbf{Acciones}             &
		\begin{enumerate}
			\def\labelenumi{\arabic{enumi}.}
			\tightlist
            \item El servidor realiza la exportación del modelo global resultante
		\end{enumerate}\\
		\textbf{Postcondición}        & Modelo global exportado \\
		\textbf{Excepciones}          & El modelo no ha finalizado el entrenamiento \\
		\textbf{Importancia}          & Alta \\
		\bottomrule
	\end{tabularx}
	\caption{CU-29 Exportar modelo global}
\end{table}

\begin{table}[p]
	\centering
	\begin{tabularx}{\linewidth}{ p{0.21\columnwidth} p{0.71\columnwidth} }
		\toprule
		\textbf{CU-30}    & \textbf{Iniciar cliente federado}\\
		\toprule
		\textbf{Versión}              & 1.0    \\
		\textbf{Autor}                & Alejandro Diez Bermejo \\
		\textbf{Requisitos asociados} & RF-5.5, RF-5.5.1, RF-5.5.2 \\
		\textbf{Descripción}          & El usuario quiere iniciar un cliente de aprendizaje federado con parámetros configurables \\
        \textbf{Precondición}         & Hay acceso a internet \\
                                      & Modelo base está cargado \\
                                      & Datasets están cargados \\
		\textbf{Acciones}             &
		\begin{enumerate}
			\def\labelenumi{\arabic{enumi}.}
			\tightlist
            \item El usuario ejecuta el script con el argumento \texttt{-c} y la dirección IP del servidor junto a su puerto
            \item El usuario indica configuraciones adicionales como \texttt{-{}-batch-size} o \texttt{-{}-epochs}
            \item El script inicializa el cliente
            \item El cliente se conecta con el servidor
		\end{enumerate}\\
		\textbf{Postcondición}        & Cliente federado inicializado \\
		\textbf{Excepciones}          & No se ha podido establecer conexión con el servidor\\
		\textbf{Importancia}          & Alta \\
		\bottomrule
	\end{tabularx}
	\caption{CU-30 Iniciar cliente federado}
\end{table}

\begin{table}[p]
	\centering
	\begin{tabularx}{\linewidth}{ p{0.21\columnwidth} p{0.71\columnwidth} }
		\toprule
		\textbf{CU-31}    & \textbf{Entrenar modelo}\\
		\toprule
		\textbf{Versión}              & 1.0    \\
		\textbf{Autor}                & Alejandro Diez Bermejo \\
		\textbf{Requisitos asociados} & RF-5.5.1 \\
		\textbf{Descripción}          & El cliente entrena el modelo que ha recibido del servidor \\
        \textbf{Precondición}         & Se ha recibido un modelo del servidor \\
                                      & Dataset de entrenamiento \\
		\textbf{Acciones}             &
		\begin{enumerate}
			\def\labelenumi{\arabic{enumi}.}
			\tightlist
            \item El cliente entrena el modelo
            \item El cliente envía el modelo al servidor junto a las métricas de entrenamiento
		\end{enumerate}\\
		\textbf{Postcondición}        & Modelo local entrenado \\
		\textbf{Excepciones}          & Error durante el entrenamiento \\
		\textbf{Importancia}          & Alta \\
		\bottomrule
	\end{tabularx}
	\caption{CU-31 Entrenar modelo}
\end{table}

\begin{table}[p]
	\centering
	\begin{tabularx}{\linewidth}{ p{0.21\columnwidth} p{0.71\columnwidth} }
		\toprule
		\textbf{CU-32}    & \textbf{Evaluar modelo}\\
		\toprule
		\textbf{Versión}              & 1.0    \\
		\textbf{Autor}                & Alejandro Diez Bermejo \\
		\textbf{Requisitos asociados} & RF-5.5.2 \\
		\textbf{Descripción}          & El cliente evalúa el modelo que ha recibido del servidor \\
        \textbf{Precondición}         & Se ha recibido un modelo del servidor \\
                                      & Dataset de validación \\
		\textbf{Acciones}             &
		\begin{enumerate}
			\def\labelenumi{\arabic{enumi}.}
			\tightlist
            \item El cliente evalúa el modelo
            \item El cliente envía las métricas de evaluación al servidor
		\end{enumerate}\\
		\textbf{Postcondición}        & Métricas de evaluación \\
		\textbf{Excepciones}          & Error durante la evaluación \\
		\textbf{Importancia}          & Alta \\
		\bottomrule
	\end{tabularx}
	\caption{CU-32 Evaluar modelo}
\end{table}

\begin{table}[p]
	\centering
	\begin{tabularx}{\linewidth}{ p{0.21\columnwidth} p{0.71\columnwidth} }
		\toprule
		\textbf{CU-33}    & \textbf{Visualizar métricas}\\
		\toprule
		\textbf{Versión}              & 1.0    \\
		\textbf{Autor}                & Alejandro Diez Bermejo \\
		\textbf{Requisitos asociados} & RF-5.6 \\
		\textbf{Descripción}          & El usuario quiere visualizar las métricas generadas durante el entrenamiento y la evaluación \\
        \textbf{Precondición}         & Existe un archivo de métricas \\
		\textbf{Acciones}             &
		\begin{enumerate}
			\def\labelenumi{\arabic{enumi}.}
			\tightlist
            \item El usuario ejecuta el script con el argumento \texttt{-m} seguido de la ruta del archivo de métricas generado por el servidor
            \item El script genera diferentes gráficas y archivos a partir del archivo de métricas
		\end{enumerate}\\
		\textbf{Postcondición}        & Archivos de valores y gráficas \\
		\textbf{Excepciones}          & Formato inválido del archivo \\
		\textbf{Importancia}          & Alta \\
		\bottomrule
	\end{tabularx}
	\caption{CU-33 Visualizar métricas}
\end{table}
