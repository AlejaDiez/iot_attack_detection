\apendice{Documentación de usuario}

\section{Introducción}
\label{sec:UsuarioIntroduccion}
En este apartado se van a detallar los requisitos que necesita el usuario para poder usar el software desarrollado y una guía detallada de como utilizar el simulador.

\section{Requisitos de usuarios}
\label{sec:RequisitosUsuario}
Para poder usar el Simulador IoT, es necesario que el usuario cumpla los siguientes requisitos:
\begin{itemize}
    \item Se requiere una \textbf{conexión a internet} para acceder a la aplicación web.
    \item Si el usuario desea crear su propia biblioteca de comandos, ataques o interceptores, debe contar con un \textbf{editor de código}.
    \item Para utilizar modelos de detección personalizados, el usuario debe tener instalado \textbf{Python} y la biblioteca \textbf{\texttt{tensorflowjs}}, además de un \textbf{editor de código} para implementar la función de análisis.
\end{itemize}

\section{Instalación}
\label{sec:InstalacionUsuario}
Para un uso básico del simulador, sin necesidad de implementar funcionalidades adicionales, no se requiere ninguna instalación. Es suficiente con contar con un navegador web compatible.

Sin embargo, si se desea utilizar una biblioteca externa o modelos de predicción personalizados, es necesario contar con un \textbf{editor de código}, así como tener instalado \textbf{Python} y la biblioteca \textbf{tensorflowjs}.

Para instalar tensorflowjs, es necesario ejecutar el siguiente comando en la terminal del sistema:
\begin{verbatim}
$ pip install tensorflowjs
\end{verbatim}

\section{Manual del usuario}
\label{sec:ManualUsuario}
En esta sección se va a realizar un recorrido completo por el simulador, presentando en detalle cada una de sus funciones y utilidades. Además, se explicará de forma detallada el proceso de creación de una biblioteca personalizada de comandos, ataques e interceptores, así como la adaptación de un modelo de inteligencia artificial desarrollado en Python para su integración en el simulador.

\subsection{Acceso al simulador}
\label{subsec:AccesoSimulador}
Para poder acceder al Simulador IoT, es necesario introducir la siguiente dirección en el navegador web:

\begin{center}
    \url{https://alejadiez.github.io/iot_attack_detection/}
\end{center}
\imagen{simulador_iot}{Logo del Simulador IoT}{0.24}

\newpage
\subsection{Vista general}
\label{subsec:VistaGeneral}
El simulador está compuesto por una barra de menús flotante en la parte superior y un lienzo desplazable en ambas direcciones, donde se diseñará la arquitectura de la red.

\imagen{vista_general}{Vista general del Simulador IoT}{1}

Para comprender mejor la vista general, hay un video explicativo disponible en el siguiente enlace: \url{https://github.com/AlejaDiez/iot_attack_detection/blob/main/docs/videos/general_view.mov}

\subsubsection{Barra de menús}
\label{subsubsec:BarraMenus}
\imagen{vista_menu}{Vista de la barra de menús}{0.6}

En esta barra de menús se encuentran todas las posibles opciones del simulador, que están divididas por diferentes secciones:
\begin{itemize}
    \item \textbf{Archivo}: en este menú se encuentran todas las opciones relacionadas con la gestión de datos del simulador.
    \begin{itemize}
        \item \textbf{Nuevo archivo}: permite crear un nuevo proyecto, eliminando el proyecto actual.
        \item \textbf{Abrir...}: permite abrir un proyecto guardado en el equipo.
        \item \textbf{Importar biblioteca externa}: permite importar una biblioteca de comandos, ataques e interceptores. Si ya existe una biblioteca importada, ofrece la opción de eliminar la actual o reemplazarla por una nueva.
        \item \textbf{Importar modelos}: permite importar modelos para el análisis de las trazas. En caso de haber modelos importados, se puede eliminar o reemplazar por otros.
        \item \textbf{Guardar}: permite guardar el estado actual del proyecto en un archivo \textit{.yaml}.
    \end{itemize}
    \item \textbf{Editar}: en este menú se encuentran todas las opciones relacionadas con la edición del proyecto actual.
    \begin{itemize}
        \item \textbf{Deshacer}: permite revertir los cambios realizados.
        \item \textbf{Rehacer}: permite restaurar los cambios deshechos.
    \end{itemize}
    \item \textbf{Insertar}: en este menú se encuentran todas las opciones relacionadas con la inserción de nodos en la red.
    \begin{itemize}
        \item \textbf{Router}: permite insertar un nodo tipo router en la red.
        \item \textbf{Dispositivo}: permite insertar un nodo tipo dispositivo en la red.
    \end{itemize}
    \item \textbf{Ver}: en este menú se encuentran todas las opciones relacionadas con la configuración de visualización del simulador.
    \begin{itemize}
        \item \textbf{Idioma}: permite cambiar el idioma del simulador. Hay disponibles el Alemán, Inglés, Español, Francés, Italiano y Portugués.
        \item \textbf{Alto contraste}: permite cambiar al modo de alto contraste para personas con dificultad visual.
        \item \textbf{Mostrar cuadrícula}: permite cambiar el estado de visibilidad de la cuadrícula del lienzo.
        \item \textbf{Centrar}: permite centrar el lienzo.
        \item \textbf{Zoom original}: permite restablecer la escala del lienzo.
        \item \textbf{Acercar}: permite aumentar la escala del lienzo.
        \item \textbf{Alejar}: permite disminuir la escala del lienzo.
    \end{itemize}
    \item \textbf{Ayuda}: en este menú se encuentran todas las opciones relacionadas con la ayuda para el simulador.
    \begin{itemize}
        \item \textbf{Código fuente}: lleva al usuario al repositorio del proyecto, donde puede buscar la documentación o publicar cualquier comentario a mejorar.
        \item \textbf{Versión del simulador}: muestra al usuario la versión actual del simulador.
    \end{itemize}
\end{itemize}

\subsubsection{Lienzo}
\label{subsubsec:Lienzo}
\imagen{vista_lienzo}{Vista del lienzo con clic derecho}{0.6}

En el lienzo se va a mostrar toda la arquitectura de la red en el proyecto actual. Al hacer clic derecho sobre el lienzo, se muestra el siguiente menú:
\begin{itemize}
    \item \textbf{Insertar router}: permite insertar un router en la posición actual.
    \item \textbf{Insertar dispositivo}: permite insertar un dispositivo en la posición actual.
\end{itemize}

\subsection{Añadir un nodo}
\label{subsec:AnadirNodo}
Para insertar un nodo, se puede hacer desde la barra de menús o desde el lienzo haciendo clic derecho. Una vez elegido el nodo a insertar router o dispositivo, se nos muestra una modal preguntando el nombre del nodo, y en caso de ser necesario el tipo de este.

\imagen{vista_anadir_nodo}{Vista de la modal para añadir un nodo}{0.6}

Al añadir un dispositivo, es obligatorio seleccionar su tipo. Se puede elegir entre dispositivo \textbf{IoT}, que corresponde a un dispositivo normal de la red, o \textbf{ordenador}, que es un dispositivo capaz de enviar ataques.

Para comprender mejor como añadir nodos, hay un video explicativo disponible en el siguiente enlace: \url{https://github.com/AlejaDiez/iot_attack_detection/blob/main/docs/videos/add_nodes.mov}

\subsection{Editar un nodo}
\label{subsec:EditarNodo}
Para editar un nodo, es necesario seleccionar el nodo e ir al apartado de ``Configuración'' del panel, o bien hacer clic derecho en el nodo y seleccionar ``Configuración''.

\imagen{vista_editar_nodo}{Vista del panel para modificar un nodo}{0.6}

En esta sección se puede configurar el nombre del nodo, modificar el tipo y eliminar el nodo.

Para comprender mejor como editar un nodo, hay un video explicativo disponible en el siguiente enlace: \url{https://github.com/AlejaDiez/iot_attack_detection/blob/main/docs/videos/edit_node.mov}

\subsection{Eliminar un nodo}
\label{subsec:EliminarNodo}
Para eliminar un nodo, es necesario seleccionar el nodo e ir al apartado de ``Configuración'' del panel y seleccionar la opción ``Eliminar'', o bien hacer clic derecho en el nodo y seleccionar ``Eliminar''.

\imagen{vista_eliminar_nodo}{Vista de la modal para eliminar un nodo}{0.6}

Para comprender mejor como eliminar un nodo, hay un video explicativo disponible en el siguiente enlace: \url{https://github.com/AlejaDiez/iot_attack_detection/blob/main/docs/videos/delete_node.mov}

\subsection{Conectar un nodo}
\label{subsec:ConectarNodo}
Para conectar un nodo, es necesario disponer de una red con un router y al menos un nodo. Se debe seleccionar el nodo e ir al apartado de ``Tráfico de red'' del panel y seleccionar la opción ``Conectar''.

\imagen{vista_conectar_nodo}{Vista del panel para conectar un nodo}{0.6}

Para comprender mejor como conectar un nodo, hay un video explicativo disponible en el siguiente enlace: \url{https://github.com/AlejaDiez/iot_attack_detection/blob/main/docs/videos/connect_node.mov}

\subsection{Editar conexión}
\label{subsec:EditarConexion}
Para editar una conexión, se debe seleccionar la conexión y se muestra una modal.

\imagen{vista_editar_conexion}{Vista de la modal para editar una conexión}{0.6}

En esta sección se puede configurar la latencia de la conexión, la variabilidad de la latencia, el ancho de banda (bytes / segundo) y el modelo de detección activado.

Para comprender mejor como editar una conexión, hay un video explicativo disponible en el siguiente enlace: \url{https://github.com/AlejaDiez/iot_attack_detection/blob/main/docs/videos/edit_connection.mov}

\subsection{Ejecutar comando}
\label{subsec:EjecutarComando}
Para ejecutar un comando, se debe seleccionar el nodo desde donde se va a ejecutar el comando e ir al apartado de ``Tráfico de red'' del panel y seleccionar el comando a ejecutar además de los objetivos que se desee enviar el comando.

\imagen{vista_comando}{Vista del panel para enviar un comando}{0.6}

Para comprender mejor como enviar un comando, hay un video explicativo disponible en el siguiente enlace: \url{https://github.com/AlejaDiez/iot_attack_detection/blob/main/docs/videos/send_command.mov}

\subsection{Visualizar paquete}
\label{subsec:VisualizarPaquete}
Para visualizar un paquete que se ha enviado, se debe seleccionar el nodo desde donde se va a visualizar el paquete e ir al apartado de ``Tráfico de red'' del panel y seleccionar el paquete a visualizar.

\imagen{vista_paquete}{Vista de la modal para visualizar un paquete}{0.6}

Para comprender mejor como visualizar un paquete, hay un video explicativo disponible en el siguiente enlace: \url{https://github.com/AlejaDiez/iot_attack_detection/blob/main/docs/videos/view_packet.mov}

\subsection{Lanzar ataque}
\label{subsec:LanzarAtaque}
Para lanzar un ataque, se debe seleccionar un nodo de tipo ordenador desde donde se va a lanzar el ataque e ir al apartado de ``Phantom Attacker'' del panel y seleccionar el ataque a lanzar además de los objetivos que se desee enviar el ataque.

\imagen{vista_ataque}{Vista del panel para enviar un ataque}{0.6}

Para comprender mejor como enviar un ataque, hay un video explicativo disponible en el siguiente enlace: \url{https://github.com/AlejaDiez/iot_attack_detection/blob/main/docs/videos/send_attack.mov}

\subsection{Detectar un ataque}
\label{subsec:DetectarAtaque}
Para detectar los ataques, se debe tener al menos una conexión con un modelo de detección activado, además de generar un flujo maligno desde uno o varios ordenadores.

\imagen{vista_detectar_ataque}{Vista de la detección de un ataque DoS}{1}

Sobre cada conexión hay un indicador que señala si el modelo está activado, además del estado de la predicción. Si el indicador está en \textbf{azul}, el modelo aún no dispone de suficientes datos para realizar una predicción. Si está en \textbf{verde}, el flujo ha sido clasificado como benigno. En cambio, si está en \textbf{rojo}, el flujo ha sido identificado como malicioso.

Para comprender mejor como detectar ataques, hay un video explicativo disponible en el siguiente enlace: \url{https://github.com/AlejaDiez/iot_attack_detection/blob/main/docs/videos/detect_attack.mov}

\subsection{Escribir una biblioteca externa para el simulador}
\label{subsec:BibliotecaExterna}
Para crear una biblioteca externa que incluya comandos, ataques e interceptores, es necesario programar un script en JavaScript que siga una estructura específica. 

\subsubsection{Comandos}
\label{subsubsec:BibliotecaExternaComandos}
Los comandos permiten generar flujos benignos entre nodos. Cada comando debe definirse como una función cuyo nombre comience con el prefijo \textbf{\texttt{cmd\_}}, seguido del identificador. La función recibe como parámetro el nodo emisor (\texttt{self}) y una o varias direcciones IP de destino (\texttt{target} / \texttt{...targets}).

\begin{verbatim}
/**
 * @name Comando
 * @param {Object} self - El nodo que ejecuta el comando
 * @param {string} self.mac - Dirección MAC
 * @param {string} self.ip - Dirección IP
 * @param {string} self.name - Nombre
 * @param {string} self.type - Tipo
 * @param {function(packet) => void} self.send - Función de envío
 * @param {string} target - Dirección IP objetivo
 * @returns {void}
 */
function cmd_Comando(self, target) {
    ...
}
\end{verbatim}

\subsubsection{Ataques}
\label{subsubsec:BibliotecaExternaAtaques}
Los ataques permiten generar flujos malignos entre un nodo ordenador y los demás nodos. Cada ataque debe definirse como una función cuyo nombre comience con el prefijo \textbf{\texttt{atk\_}}, seguido del identificador. La función recibe como parámetro el nodo atacante (\texttt{self}) y una o varias direcciones IP de destino (\texttt{target} / \texttt{...targets}).

\begin{verbatim}
/**
 * @name Ataque Multiple
 * @param {Object} self - El nodo que lanza el ataque
 * @param {string} self.mac - Dirección MAC
 * @param {string} self.ip - Dirección IP
 * @param {string} self.name - Nombre
 * @param {string} self.type - Tipo
 * @param {function(packet) => void} self.send - Función de envío
 * @param {...string} targets - Direcciones IP objetivos
 * @returns {void}
 */
function atk_Ataque_Multiple(self, ...targets) {
    ...
}
\end{verbatim}

\subsubsection{Interceptores}
\label{subsubsec:BibliotecaExternaInterceptores}
Los interceptores permiten reaccionar a paquetes que llegan a un nodo. Para definir un interceptor debe de ser una función cuyo nombre sea:
\begin{itemize}
    \item \textbf{\texttt{intcp}}: interceptor genérico
    \item \textbf{\texttt{intcp\_router}}: interceptor para el nodo de tipo router
    \item \textbf{\texttt{intcp\_iot}}: interceptor para el nodo de tipo iot
    \item \textbf{\texttt{intcp\_computer}}: interceptor para el nodo de tipo computer
\end{itemize}
La función recibe como parámetro el nodo interceptor (\texttt{self}) y el paquete que se ha interceptado (\texttt{packet}). En caso de que se desee descartar la ejecución del interceptor predeterminado, es necesario devolver \texttt{null} u otro valor.
\begin{verbatim}
/**
 * @param {Object} self - El nodo que lanza el ataque
 * @param {string} self.mac - Dirección MAC
 * @param {string} self.ip - Dirección IP
 * @param {string} self.name - Nombre
 * @param {string} self.type - Tipo
 * @param {function(packet) => void} self.send - Función de envío
 * @param {Object} packet - Paquete interceptado
 * @param {string} packet.srcIP - Dirección IP de origen
 * @param {number} [packet.srcPort] - Puerto de origen
 * @param {string} packet.dstIP - Dirección IP de destino
 * @param {number} [packet.dstPort] - Puerto de destino
 * @param {number} packet.transportProtocol - Protocolo de transporte
 * @param {number} [packet.applicationProtocol] - Protocolo de aplicación
 * @param {string} [packet.payload] - Carga útil
 * @param {number} packet.totalBytes - Tamaño total en bytes
 * @param {number} packet.headerSize - Tamaño de la cabecera
 * @param {number} packet.payloadSize - Tamaño de la carga útil
 * @param {Date} packet.timestamp - Marca temporal
 * @param {number} packet.ttl - Tiempo de vida (TTL)
 * @param {number} packet.type - Tipo de mensaje ICMP
 * @param {number} packet.code - Código de mensaje ICMP
 * @param {number} packet.identifier - Identificador del mensaje ICMP
 * @param {number} packet.sequence - Número de secuencia del mensaje ICMP o TCP
 * @param {number} packet.tcpFlags - Banderas TCP
 * @param {number} packet.ack - Número de confirmación
 * @returns {void} Retornar un valor, cancela el interceptor predeterminado
 */
function intcp(self, packet) {
    ...
}
\end{verbatim}

Para comprender mejor como escribir una biblioteca externa, hay un video explicativo disponible en el siguiente enlace: \url{https://github.com/AlejaDiez/iot_attack_detection/blob/main/docs/videos/library.mov}

\subsection{Adaptar un modelo de inteligencia artificial}
\label{subsec:AdaptarModelo}
Para adaptar un modelo de inteligencia artificial desarrollado con \textbf{TensorFlow} en Python al simulador, es necesario convertirlo a un formato compatible con \textbf{TensorFlow.js} y crear un \textbf{script} que utilice el modelo para predecir. Estos archivos se comprimirán en una carpeta \textbf{\texttt{.zip}} que será lo que interpretará el simulador.

\subsubsection{Convertir modelo}
\label{subsubsec:AdaptarModeloConvertir}
Para adaptar \cite{tensorflowjs} el modelo es necesario tener instalada la biblioteca \texttt{tensorflowjs} y tener un modelo\footnote{El modelo debe haber sido entrenado con una versión de Keras v2. En caso contrario, es necesario convertirlo para que sea compatible con esta versión.} en formato \texttt{.h5}.

Para realizar la conversión, es necesario ejecutar el siguiente comando en la terminal del sistema:
\begin{verbatim}
$ tensorflowjs_converter --input_format=keras \
        /ruta/al/modelo_guardado \
        /ruta/de/salida
\end{verbatim}
Este proceso realiza una conversión a un modelo compatible con \textbf{TensorFlow.js}, generando un archivo \texttt{.json} con la configuración del modelo y uno o varios archivos \texttt{.bin} que contienen los pesos.

\subsubsection{Script de análisis}
\label{subsubsec:AdaptarModeloScript}
Para utilizar el modelo, es necesario crear un archivo JavaScript que contenga una función llamada \textbf{\texttt{analyze}}. Esta función se encarga de analizar cada uno de los paquetes que se le pasen como parámetro y devolver un valor según el resultado:
\begin{itemize}
    \item \textbf{\texttt{true}}: se ha detectado un ataque
    \item \textbf{\texttt{false}}: no se ha detectado ataque
    \item \textbf{\texttt{any}}: el modelo no dispone de datos suficientes para realizar un análisis
\end{itemize}

Cabe destacar que el script dispone de dos variables globales:
\begin{itemize}
    \item \textbf{\texttt{tf}}: biblioteca de TensorFlow.js
    \item \textbf{\texttt{model}}: modelo de predicción a usar
\end{itemize}

\begin{verbatim}
/**
 * @param {Object} packet - Paquete interceptado
 * @param {string} packet.srcIP - Dirección IP de origen
 * @param {number} [packet.srcPort] - Puerto de origen
 * @param {string} packet.dstIP - Dirección IP de destino
 * @param {number} [packet.dstPort] - Puerto de destino
 * @param {number} packet.transportProtocol - Protocolo de transporte
 * @param {number} [packet.applicationProtocol] - Protocolo de aplicación
 * @param {string} [packet.payload] - Carga útil
 * @param {number} packet.totalBytes - Tamaño total en bytes
 * @param {number} packet.headerSize - Tamaño de la cabecera
 * @param {number} packet.payloadSize - Tamaño de la carga útil
 * @param {Date} packet.timestamp - Marca temporal
 * @param {number} packet.ttl - Tiempo de vida (TTL)
 * @param {number} packet.type - Tipo de mensaje ICMP
 * @param {number} packet.code - Código de mensaje ICMP
 * @param {number} packet.identifier - Identificador del mensaje ICMP
 * @param {number} packet.sequence - Número de secuencia del mensaje ICMP o TCP
 * @param {number} packet.tcpFlags - Banderas TCP
 * @param {number} packet.ack - Número de confirmación
 * @returns {boolean | void} - true: ataque, false: benigno, void: esperando
 */
function analyze(packet) {
    ...
}
\end{verbatim}

Para comprender mejor como adaptar un modelo de inteligencia artificial, hay un video explicativo disponible en el siguiente enlace: \url{https://github.com/AlejaDiez/iot_attack_detection/blob/main/docs/videos/model.mov}
